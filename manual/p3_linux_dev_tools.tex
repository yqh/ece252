\chapter{Introduction to ECE Linux Programming Environment}
\label{ch_linux_env}

\section{ECE Linux Servers}

There are a group of Linux Ubuntu servers that are open to ECE undergraduate students. The machines are listed at url: \url{https://ece.uwaterloo.ca/Nexus/arbeau/clients}. To access one of the machines, we recommend to use the alias name of \verb+eceubuntu.uwaterloo.ca+, which will direct the user to the most lightly loaded machine at the time of login.

To access these machines from off campus. One way is to use the \href{https://uwaterloo.ca/information-systems-technology/services/virtual-private-network-vpn}{campus VPN}.
Another way is to first connect to \verb+eceterm.uwaterloo.ca+ and then connect to other Linux servers from there. Note that the \verb+eceterm+ should not be used for computing jobs, it is for accessing other Linux servers on campus.

\section{Connecting to Linux servers}

A terminal client software that supports secure shell (ssh) will allow you to remotely connect to the Linux servers. MobaXterm is a convenient application that not only supports ssh, but also has a built-in X server that allows one to run Graphical User Interface (GUI) applications from the Linux servers.
\begin{figure}[!htb]
  \centering
  \includegraphics[width=5.5in]{img/lab0/MobaXterm/MobaXterm_Path}
  \caption{MobaXterm Path on ECE Nexus Windows 10 Machines.}
  \label{fig_MobaXterm_Path}
\end{figure}

\begin{figure}[!htb]
  \centering
  \includegraphics[width=6in]{img/lab0/MobaXterm/MobaXterm_Start_Page}
  \caption{MobaXterm Welcome Page.}
  \label{fig_MobaXterm_Start_Page}
\end{figure}

Use the File Explorer to navigate to \verb+Q:\eng\ece\Util+ folder, scroll down until you find the MobaXterm icon and double click it (see Figure \ref{fig_MobaXterm_Path}). The MobaXterm window will pop up. There is a grey rectangular button labelled ``Start local terminal'' in the middle (see Figure \ref{fig_MobaXterm_Start_Page}). Click this button.

Then a terminal session starts. You will need to use the command line \verb+ssh+ command to connect to the Linux server. Use your UWID and password to login. The syntax of the command is as follows:
%\begin{verbatim}
\begin{lstlisting}[style=bash]
ssh -XY <UWID>@eceubuntu.uwaterloo.ca
\end{lstlisting}
% \end{verbatim}
See Figure \ref{fig_MobaXterm_Login_Password} for reference.
\begin{figure}[!htb]
  \centering
  \includegraphics[width=6in]{img/lab0/MobaXterm/MobaXterm_Login_Password}
  \caption{MobaXterm Welcome Page.}
  \label{fig_MobaXterm_Login_Password}
\end{figure}


All your ECE Linux account files are accessible through P Drive on Nexus machines (See Figure \ref{fig_lab0_P_Drive}). The P Drive is only accessible within campus network. Mapping the Linux account as a network drive off campus is not supported due to security reasons.
\begin{figure}[!htb]
  \centering
  \includegraphics[width=6in]{img/lab0/lab0_P_Drive}
  \caption{Linux files on P drive, a network mapped drive.}
  \label{fig_lab0_P_Drive}
\end{figure}

\section{Basic Software Development Tools}

%[NOTE: adding a flow chart here]
To develop a program, there are three important steps.
First, a program is started from source code written by programmers. Second, the source code is then compiled into object code, which is a binary. Non-trivial project normally contains more than one source file. Each source file is compiled into one object code and the linker would finally link all the object code to generate the final target, which is the executable that runs. The steps of compiling and linking are also known as building a target. It is very rare that the target will run perfectly the first time it is built. Most of time we need to fix defects and bugs in the code and the this is the third step. The debugger is a tool to help you identify the bug and fix it. Table \ref{tb_prog_tools} shows the key steps in programming work flow and example tools provided by a general purpose Linux operating system.

\begin{table}[ht]
\begin{center}
%\footnotesize{
\begin{tabular}{llll}
\hline
Task & Tool & Examples\\ \hline
Editing the source code &    Editor & vi, emacs\\  
Compiling the source code &  Compiler & gcc \\ 
Debugging the program	  &  Debugger & gdb, ddd \\ \hline 
\end{tabular}
%}
\caption{Programming Steps and Tools}
\label{tb_prog_tools}
\end{center}
\end{table}

Most of you probably are more familiar with a certain Integrated Development Environment (IDE) which integrates all these tools into a single environment. For example Eclipse and Visual Studio. A different approach is to select a tool in each programming step and build your own tool chain. Many seasoned Linux programmers build their own tool chains. A few popular tools are introduced in the following subsections.

\subsection{Editor}

Some editors are designed to better suit programmers' needs than others. The {\em vi} ({\em vim} and {\em gvim} belong to the vi family) and {\em emacs} ({\em xemacs} belongs to emacs family) are the two most popular editors for programming purposes.

%This manual is not meant to teach you how to use these editors. \cite{} are good references to get yourself started with these editors.

Two simple notepad editors {\em pico} and {\em nano} are also available for a simple editing job. These editors are not designed for programming activity. To use one of them to write your first {\em Hello World} program is fine though. 

After you finish editing the C source code, give the file name an extension of .c. Listing \ref{lst_helloworld} is the source code of printing "Hello World!" to the screen. 

\lstinputlisting[language=C, caption={HelloWorld C source Code}, label=lst_helloworld]{code/linux/HelloWorld/main.c}

\subsection{C Compiler}
The source code then gets fed into a compiler to be come an executable program.
The GNU project C and C++ compiler is {\em gcc}. To compile the HelloWorld source code in Listing \ref{lst_helloworld}, type the following command at the prompt:
\begin{lstlisting}[style=bash]
gcc helloworld.c+ 
\end{lstlisting}
You will notice that a new file named \verb+a.out+ is generated.
This is the executable generated from the source code. To run it, type the following command at the prompt and hit Enter.
\begin{lstlisting}[style=bash]
./a.out
\end{lstlisting}
The result is``Hello World!" appearing on the screen.

You can also instruct the compiler to name the executable another name instead of the default \verb+a.out+. The \verb+-o+ option in gcc allows one to name the executable a name. For example, the following command will generate an executable named ``\verb+helloworld.out+".
\begin{lstlisting}[style=bash]
gcc helloworld.c -o helloworld.out
\end{lstlisting}
although there is no requirement that the name ends in \verb+.out+.
\subsection{Debugger}
The GNU debugger gdb is a command line debugger. Many GUI debugger uses gdb as the back-end engine. One GNU GUI debugger is {\em ddd}. It has a powerful data display functionality. 

GDB needs to read debugging information from the binary in order to be able to help one to debug the code. The \verb+-g+ option in gcc tells the compiler to produce such debugging information in the generated executable. In order to use gdb to debug our simple HelloWorld program, we need to compile it with the following command:
\begin{lstlisting}[style=bash]
gcc -g helloworld.c -o helloworld.out
\end{lstlisting}

The following command calls gdb to debug the helloworld.out 
\begin{lstlisting}[style=bash]
gdb helloworld.out 
\end{lstlisting}
This starts a gdb session. At the \verb+(gdb)+ prompt, you can issue gdb command such as \verb+b main+ to set up a break point at the entry point of main function. The \verb+l+ lists source code. The \verb+n+ steps to the next statement in the same function. The \verb+s+ steps into a function. The \verb+p+ prints a variable value provided you supply the name of the variable. Type \verb+h+ to see more gdb commands.

Compared to gdb command line interface, the ddd GUI interface is more user friendly and easy to use. To start a ddd session,
type the command
\begin{lstlisting}[style=bash]
ddd 
\end{lstlisting}
and click File $\rightarrow$ Open Program to open an executable such as helloworld.out. You will then see gdb console in the bottom window with the source window on top of the gdb console window. You could see the value of variables of the program through the data window, which is on top of the source code window. Select View to toggle all these three windows. 

A good gdb/ddd tutorial URL is \url{https://www.cs.swarthmore.edu/~newhall/unixhelp/howto_gdb.php}. Pay special attention to the section on how to trouble shoot segmentation fault problems.
\section{More on Development Tools}

For any non-trivial software project, it normally contains multiple source code files. Developers need tools to manage the project build process. Also project normally are done by several developers. A version control tool is also needed.

\subsection{How to Automate Build}

Make is an utility to automate the build process. Compilation is a cpu-intensive job and one only wants to re-compile the file that has been changed when you build a target instead of re-compile all source file regardless. 
The \verb+make+ utility uses a Makefile to specify the dependency of object files and automatically recompile files that has been modified after the last target is built. 

In a Makefile, one specifies the targets to be built, what prerequisites the target depends on and what commands are used to build the target given these prerequisites. These are the {\em rules} contained in Makefile. The Makefile has its own syntax.
The general form of a Makefile rule is:
\begin{lstlisting}[style=makefile]
target ...: prerequisites ... 
        recipe
        ...	
        ...	
\end{lstlisting}
One important note is that each recipe line starts with a TAB key rather than white spaces. To build a target, use command \verb+make+ followed by the target name or omit the target name to default to the first target in the Makefile. For example
\begin{lstlisting}[style=bash]
make
\end{lstlisting}
will build your lab starter code.

Listing \ref{lst_makefile_hello} is our first attempt to write a very simple Makefile.

\lstinputlisting[
    language=make, 
    caption={Hello World Makefile: First Attempt}, 
    style=makefile,
    label=lst_makefile_hello
]{code/linux/HelloWorld/Makefile1}
The following command will generate the \verb+helloworld.out+ executable file.
\begin{lstlisting}[style=bash]
make helloworld.out
\end{lstlisting}

Our second attempt is to break the single line gcc command into two steps. First is to {\em compile} the source code into object code .o file. Second is to {\em link} the object code to one final executable binary. 
Listing \ref{lst_makefile_hello2} is our second attempted  version of Makefile.

\lstinputlisting[
    language=make, 
    caption={Hello World Makefile: Second Attempt}, 
    style=makefile,
    label=lst_makefile_hello2
]{code/linux/HelloWorld/Makefile2}

When a project contains multiple files, separating object code compilation and linking stages would give a clear dependency relationship among code. Assume that we now need to build a project that contains two source files \verb+src1.c+ and \verb+src2.c+ and we want the final executable to be named as \verb+app.out+.
Listing \ref{lst_makefile3} is a typical example Makefile that is closer to what you will see in the real world.
\lstinputlisting[
    language=make, 
    caption={A More Real Makefile: First Attempt}, 
    style=makefile,
    label=lst_makefile3
]{code/linux/HelloWorld/Makefile3}
We also have added a target named {\em clean} so that \verb+make clean+ will clean the build.

So far we have seen the Makefile contains {\em explicit rules}. Makefile can also contain {\em implicit rules}, {\em variable definitions}, {\em directives} and
{\em comments}. 
Listing \ref{lst_makefile4} is a Makefile that is used in the real world.
\lstinputlisting[
    language=make, 
    numbers=left,
    numberstyle=\tiny,
    stepnumber=1,
    numbersep=5pt,
    caption={A Real World Makefile}, 
    style=makefile,
    label=lst_makefile4
]{code/linux/HelloWorld/Makefile4}
Line $1$ is a comment. Lines $2-7$ are variable definitions. Line $12$ is an implicit rule to generate .o file for each .c file. 
See \url{http://www.gnu.org/software/make/manual/make.html} to explore more of makefile.

\subsection{Version Control Software}
We use Git version control software. It is installed both on the Linux servers and Nexus windows machines. If you decide to use GitHub to host your repository, please make sure it is a {\em private} one. Go to \url{http://github.com/edu} to see how to obtain five private repositories for two years on GitHub for free. 
%This [] is a good reference on how to use Git and Dropbox together to create private repositories.

%Table \ref{} lists some frequently used git commands.

\subsection{Integrated Development Environment}
Visual Studio Code has been installed on all ECE ubuntu servers. Use the following command to invoke it. Note you will need X Windows support for the GUI interface.
\begin{lstlisting}[style=bash]
/snap/bin/code
\end{lstlisting}
\iffalse
Eclipse with C/C++ Plug-in has been installed on all ECE Linux servers. Type the following command to bring up the eclipse frontend.
\begin{lstlisting}[style=bash]
/opt/eclipse64/eclipse
\end{lstlisting}
This eclipse is not the same as the default eclipse under \verb+/usr/bin+ directory. You may find running eclipse over network performs poorly at home though. It depends on how fast your network speed is. 

If you have Linux operating system installed on your own personal computer, then you can download the eclipse with C/C++ plugin from the eclipse web site and then run it from your own local computer. However you should always make sure the program will also work on ecelinux machines, which is the environment TAs would be using to test your code.
\fi
\section{Man Page}
Linux provides manual pages. You can use the command \verb+man+ followed by the specific command or function you are interested in to obtain detailed information. 

Mange pages are grouped into sections. We list frequently used sections here: 
\begin{itemize}
\item Section 1 contains user commands.
\item Section 2 contains system calls
\item Section 3 contains library functions
\item Section 7 covers conventions and miscellany. 
\end{itemize}

To specify which section you want to see, provide the section number after the \verb+man+ command. For example,
\begin{lstlisting}[style=bash]
man 2 stat
\end{lstlisting}
shows the system call \verb+stat+ man page. If you omit the \verb+2+ in the command, then it will return the command \verb+stat+ man page.

You can also use \verb+man -k+  or \verb+apropos+ followed by a string to obtain a list of man pages that contain the string. The Whatis database is searched and now run \verb+man whatis+ to see more details of \verb+whatis+.

%%% Local Variables:
%%% mode: latex
%%% TeX-master: "main_book"
%%% End:
